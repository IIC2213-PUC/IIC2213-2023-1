\begin{enumerate}
	\item[(a)] Sea $P$ un conjunto de variables proposicionales. Dado $\Sigma\subseteq\mathcal{L}(P)$ y fórmulas $\alpha,\beta\in\mathcal{L}(P)$ demuestre que \vspace{-3ex}
	\begin{center}
		$\Sigma\models(\alpha\rightarrow\beta)$\quad si y solo si\quad $\Sigma\cup\{\alpha\}\models\beta$
	\end{center}
	\item[(b)] Sea $\mathcal{L}=\{R\}$ vocabulario con $R$ símbolo de relación $n$-aria y sea $\mathfrak{A}$ una $\mathcal{L}$-estructura con dominio finito. Construya una fórmula proposicional $\varphi$ tal que $\varphi$ sea satisfacible si y solo si existe un automorfismo no trivial en $\mathfrak{A}$. Demuestre la correctitud de su construcción.
	\textit{Aclaración:} un automorfismo no trivial es un isomorfismo de $\mathfrak{A}$ en $\mathfrak{A}$, tal que es distinto de la función $f(a)=a$.

\end{enumerate}
