\documentclass{article}
\usepackage[utf8]{inputenc}
\usepackage[english, spanish]{babel}
%\usepackage[dvips]{graphics}
\usepackage{amsmath}
\usepackage{amssymb}
\usepackage{fullpage}
\usepackage{epsfig}
\usepackage{multicol}
\usepackage{wasysym}
\usepackage{tikz}
\usepackage{graphicx}
\usepackage{mathtools}
\usepackage{enumitem}

\usetikzlibrary{arrows,decorations.pathmorphing,backgrounds,positioning,fit,calc,automata}
\usetikzlibrary{decorations.pathreplacing}
\usetikzlibrary{trees}
\usetikzlibrary{shapes}
\usetikzlibrary{chains}
\usetikzlibrary{patterns}
\usetikzlibrary{matrix}
%\usepackage{stackengine}


%\usepackage[usenames,dvipsnames]{xcolor}
%\usepackage{hyperref}
%\definecolor{linkcolour}{rgb}{0,0.2,0.6}
%\hypersetup{colorlinks,breaklinks,urlcolor=linkcolour,linkcolor=linkcolour}

\parindent 0pt
%\parskip 0pt

\newcommand{\cL}{\mathcal{L}}

\begin{document}

\includegraphics[width=2cm]{uc.png}
\vspace*{-1.9cm}

\hspace*{2.1cm}
 \begin{tabular}{l}
  \sc Pontificia Universidad Católica de Chile \\
  \sc Escuela de Ingeniería \\
  \sc Departamento de Ciencia de la Computación
  %\vspace{15\baselineskip}\mbox{}
  %\vspace{-3mm}\mbox{}
 \end{tabular}
 \bigskip

\vspace*{5mm}
\begin{center}
{IIC2213 --- Lógica para ciencia de la computación --- 1' 2023} \\
\vspace{3mm}
{\Large\bf TAREA 1} \\
\vspace{2mm}
\end{center}

\begin{tabular}{ll}
Publicación: & Martes 14 de marzo. \\
Entrega: & \textbf{Lunes 27 de marzo hasta las 23:59 horas}. \\
\end{tabular}

\subsection*{Indicaciones}

\begin{itemize}
\item Cada pregunta tiene 6 puntos ($+1$ base) y la nota de la tarea es el promedio de  las preguntas.
\item Todos los incisos de una pregunta pregunta tienen el mismo puntaje.
\item Cada solución debe estar escrita en \LaTeX. No se aceptarán tareas escritas de otra forma.
\item La tarea es individual, pudiendo discutirla con sus pares. Toda referencia externa debe citarse.
\end{itemize}

\subsection*{Objetivos}
Esta tarea busca evaluar si el estudiante es capaz de
\begin{itemize}
    \item Definir inductivamente propiedades/objetos en $\cL(P)$
    \item Demostrar usando diversas técnicas, entre ellas inducción en $\cL(P)$
    \item Aplicar conceptos de sintaxis y semántica proposicional, y consecuencia lógica
\end{itemize}

\subsection*{Pregunta 1: Sintaxis e inducción en $\cL(P)$}
Decimos que dos grafos $G_1 = (V_1, E_1)$ y $G_2 =
(V_2, E_2)$\ son {\em isomorfos}, denotado por $G_1\cong G_2$, si existe una biyección $f : V_1
\rightarrow V_2$ tal que para todo $\{a,b\}\in E_1$ se tiene que
$$\{a,b\} \in E_1\quad\text{si y solo si}\quad\{f(a), f(b)\} \in E_2$$ 

Dado un par de grafos no dirigidos $G_1=(V_1,E_1)$ y $G_2=(V_2,E_2)$, construya una fórmula proposicional $\varphi_\text{iso}$ tal que 
$$\varphi_\text{iso}\text{ es satisfacible\quad si y solo si}\quad G_1\cong G_2$$
Demuestre la correctitud de su construcción.

\subsubsection*{Solución P1.}
\input{sol1}

\newpage
\subsection*{Pregunta 2: Semántica en $\cL(P)$ y equivalencia lógica}
En un grafo dirigido $G=(N,A)$, un camino dirigido de $n$ nodos es una secuencia $u_0,\ldots,u_n$ tal que
\begin{itemize}
    \item $u_i\in N$, para todo $0\leq i\leq n$
    \item $(u_i,u_{i+1})\in A$, para todo $0\leq i < n$
\end{itemize}
Un ciclo simple dirigido de $n$ nodos es un camino dirigido tal que $u_0=u_n$ y todos los demás nodos deben ser distintos.\medskip

Sea el vocabulario $\cL=\{E\}$ con símbolo de relación binaria $E$. Construya $\cL$-oraciones en lógica de primer orden que representen las siguientes propiedades. 
\begin{enumerate}
    \item[(a)] El grafo es un clique. Utilice para esto la noción de clique dirigido en que deben existir todas las aristas dirigidas posibles.
    \item[(b)] El grafo contiene un ciclo simple dirigido de 3 nodos.
\end{enumerate}
Observe que cada oración deben ser satisfecha por una $\cL$-estructura $\mathfrak{A}$ si, y solo si, el grafo representado por $\mathfrak{A}$ cumple la propiedad modelada con dicha oración.

\subsubsection*{Solución P2.}
\input{sol2}

\newpage
\subsection*{Pregunta 3: Consecuencia lógica en $\cL(P)$}
Construya una máquina de Turing determinista con una cinta que se detenga en todo input y acepte el lenguaje 
$$L=\{w\in \{0,1\}^*\mid w\text{ tiene la misma cantidad de símbolos }0\text{ y }1\}$$
\textbf{Formato de entrega.} Su respuesta debe consistir en un archivo de texto plano \texttt{.txt} que se pueda ejecutar en \href{https://turingmachinesimulator.com/}{Turing Machine Simulator}. El formato usado en esta plataforma sigue las convenciones vistas en clase (estructura de la función parcial de transiciones) y además el simulador le ayudará a testear su propuesta. La solución se probará con 6 inputs de diferentes tamaños y la nota será la cantidad de tests aprobados $+1$.

\subsubsection*{Solución P3.}
\input{sol3}


\end{document}


